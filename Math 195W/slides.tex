\documentclass{beamer}
\usepackage{commands, pgfplots, subcaption} % Required for inserting images
\usetheme{Singapore}
\newcommand{\V}{\mathbb V}
\newcommand{\I}{\mathbb I}
\title{Illustrating Hilbert's Nullstellensatz}
\author{Timothy Cho}
\institute{UC Irvine, MATH 195}
\date{30 November 2023}
\begin{document}
\maketitle
\begin{frame}{Motivation: Why do Algebraic Geometry?}
\pause
\begin{itemize}
    \item It's cool \pause
    \item Historical: shapes/patterns $\implies$ equations \pause
    \item Algebraic geometry: shapes/patterns $\iff$ equations \pause
    \item Connections: analysis, topology, commutative algebra, etc.
\end{itemize}
\end{frame}

\begin{frame}{Roots of Polynomials}
Find the real roots of $(x^2 + 4y^2 - 4)^2 \in \R[x,y]$. \pause

$$(x^2 + 4y^2 - 4)^2 = 0 \implies x^2 + 4y^2 - 4 = 0 \implies x^2 + 4y^2 = 4$$
\pause
\begin{center}
\begin{tikzpicture}
    \draw[->] (-2.5, 0) -- (2.5,0);
    \draw[->] (0, -2) -- (0, 2);
    \draw[very thick] (0, 0) ellipse (1 and 0.5);
    \draw (1,0) node[anchor=north]{\,\,1};
    \draw (0, 0.4) node[anchor=south]{\footnotesize{\,\,\,\,\,\,\,\,\,\,1/2}};
\end{tikzpicture}
\end{center}
\end{frame}
\begin{frame}{Algebraic Varieties}
\begin{definition}
Let $k$ be a field, and let $S\subseteq k[x_1, \ldots, x_n]$. The \textit{algebraic variety} of $S$ is the set
$$\V(S) := \{x\in k^n: f(x) = 0 \textsf{ for all } f\in S\}.$$

Similarly, $V\subseteq k^n$ is an \textit{algebraic set} if $V = \V(S)$ for some $S\subseteq k[x_1,\ldots, x_n]$.
\end{definition}
\pause

Advantage: points $\iff$ pictures $\iff$ shapes

Disadvantages: hard to work with
\end{frame}
\begin{frame}{Another Example}
Let $S = \{(x^2+y^2-4)(x-1), (x^2+y^2-4)(y-1)\}\subseteq \R[x,y]$. What is $\V(S)\subseteq \R^2$? \pause
$$\begin{cases}
(x^2+y^2-4)(x-1) = 0 \\
(x^2+y^2-4)(y-1) = 0
\end{cases}$$\pause
$$\implies 0= (x^2+y^2-4)(x-1) = (x^2+y^2-4)(y-1)$$
$$\implies x^2+y^2 = 4 \textsf{ or } (x,y) = (1,1)$$
\end{frame}
\begin{frame}{Another Example}
\begin{figure}[h!]
    \centering
    \begin{tikzpicture}
    \draw[->] (-3,0) -- (3,0);
    \draw[->] (0,-3) -- (0,3);
    \draw[very thick] (0,0) circle (2);
    \filldraw[very thick] (1,1) circle (0.05);
    \draw \xtick{1}; \draw \xtick{2};
    \draw \ytick 1; \draw \ytick 2;
    \draw \xtick{-1}; \draw \xtick{-2};
    \draw\ytick{-1}; \draw\ytick{-2};
    \draw (1, -0.2) node[anchor=north]{1};
    \end{tikzpicture}
    \caption{$\V\Big((x^2+y^2-4)(x-1), (x^2+ y^2-4)(y-1)\Big)$}
    \label{myfigure}
\end{figure}
\end{frame}
\begin{frame}{Ideals}
Motivation: ``the other way around" \pause

\begin{definition}
Let $k$ be a field, and let $V\subseteq k^n$. The \textit{ideal} of $V$ to be the set
$$\I(V) := \{f\in k[x_1,\ldots, x_n]: f(x) = 0 \textsf{ for all } x\in V\}.$$
\end{definition}\pause
\begin{block}{Proposition}
Let $k$ be a field, and let $V\subseteq k^n$. Then $\I(V)$ is an ideal in $k[x_1, \ldots, x_n]$.
\end{block}
\pause\vspace{0.2 cm}

Advantage: we understand ideals!
\end{frame}
\begin{frame}{Computing Ideals}
\begin{columns}
\begin{column}{0.48\textwidth}
\begin{figure}[h!]
\centering
\begin{tikzpicture}[domain=0:2, scale = 0.75]
  \draw[black, line width = 0.50mm]   plot[smooth,domain=-2.5:2.5] (\x, 4 - \x*\x);
  \draw[->] (-3,0) -- (3,0);
  \draw[->] (0,-3) -- (0,4.5);
  \draw (0,4.2) node[anchor=west]{4};
  \draw \ytick 1; \draw\ytick 2;\draw\ytick 3; \draw\ytick{-1}; \draw\ytick{-2}; \draw\ytick{-3};
  \draw\xtick{1};\draw\xtick{-1};
  \draw (1.8,0) node[anchor=north]{2};
\end{tikzpicture}
\caption{$V\subseteq\R^2$}
\end{figure}
\end{column}
\begin{column}{0.48\textwidth}
What is $\I(V)$? \pause

$$V = \{(x,y)\in\R^2: x^2+y-4=0\}.$$ \pause

$f(x,y) = g(x,y)(x^2+y-4)$ vanishes for all $g(x,y)\in \R[x,y]$ \pause

$$\I(V) = \gen{x^2+y-4}.$$
\end{column}
\end{columns}
\end{frame}
\begin{frame}{$\V$ versus $\I$}
Let $k$ be a field.
\begin{center}
\def\arraystretch{1.2}
\begin{tabular}{c|c|c}
Attribute  &  $\V$ & $\I$  \\\hline \pause
Input & $S\subseteq k[x_1,\ldots, x_n]$ & $V\subseteq k^n$ \\ \pause
Output & $\V(S)\subseteq k^n$ & $\I(V) \furina k[x_1, \ldots, x_n]$ \\ \pause
Advantages & Very visual & Easy to manipulate \\ \pause
Disadvantages & Hard to manipulate & Not visual \\
\end{tabular}
\end{center}\pause
In an ideal world: $\I$ and $\V$ are mutually inverse
\end{frame}
\begin{frame}{$\V(\I(V)) = V$}
\begin{block}{Proposition}
Let $V\subseteq k^n$ be an algebraic set. Then $\V(\I(V)) = V$. That is, $\V$ is a left-inverse to $\I$.
\end{block}
\begin{proof}
Definition-chase.
\end{proof}
\end{frame}
\begin{frame}{$\V(\I(V)) = \V$: Example}
\begin{columns}
\begin{column}{0.48\textwidth}
\begin{figure}[h!]
\centering
\begin{tikzpicture}[domain=0:2, scale = 0.75]
  \draw[black, line width = 0.50mm]   plot[smooth,domain=-2.5:2.5] (\x, 4 - \x*\x);
  \draw[->] (-3,0) -- (3,0);
  \draw[->] (0,-3) -- (0,4.5);
  \draw (0,4.2) node[anchor=west]{4};
  \draw \ytick 1; \draw\ytick 2;\draw\ytick 3; \draw\ytick{-1}; \draw\ytick{-2}; \draw\ytick{-3};
  \draw\xtick{1};\draw\xtick{-1};
  \draw (1.8,0) node[anchor=north]{2};
\end{tikzpicture}
\caption{$V\subseteq\R^2$}
\end{figure}
\end{column}
\begin{column}{0.48\textwidth}
Know: $I:=\I(V) = \gen{x^2+y-4}$. \\

$\V(I)$: points in $\R^2$ that vanish on all of $I$ \pause --- but this is just $V$.
\end{column}
\end{columns}
\end{frame}
\begin{frame}{$\I(\V(I))= I$?}
\begin{columns}
\begin{column}{0.48\textwidth}
\begin{figure}[h!]
\centering
\begin{tikzpicture}
    \draw[->] (-2.5, 0) -- (2.5,0);
    \draw[->] (0, -2) -- (0, 2);
    \draw[very thick] (0, 0) ellipse (1 and 0.5);
    \draw (1,0) node[anchor=north]{\,\,1};
    \draw (0, 0.4) node[anchor=south]{\footnotesize{\,\,\,\,\,\,\,\,\,\,1/2}};
\end{tikzpicture}
\caption{$\V(I)$}
\end{figure}
\end{column}
\begin{column}{0.48\textwidth}
Let $I:= \gen{(x^2+4y^2-4)^2}$. \pause

$\I(\V(I))$: polynomials in $\R[x,y]$ that vanish on $\V(I)$

$$\I(\V(I)) = \gen{x^2+4y^2-4} \neq I.$$ \pause

But how are $\I(\V(I))$ and $I$ related?
\end{column}
\end{columns}
\end{frame}
\begin{frame}{Radical of an Ideal}
``Set of $n$th roots" \pause
\begin{definition}
Let $I\subseteq R$ be an ideal. Then the \textit{radical} of $I$ is the set
$$\sqrt I:= \{r\in R: r^n\in I \textsf{ for some } n\in\Z^+\}.$$
Similarly, we say that an ideal $I$ \textit{is radical} if $I = \sqrt I$.
\end{definition}\pause
\begin{block}{Proposition}
The radical of an ideal is an ideal.
\end{block}
\begin{block}{Proposition}
Prime ideals are radical.
\end{block}
\end{frame}
\begin{frame}{A Rad(ical) Example}
\begin{columns}
\begin{column}{0.48\textwidth}
\begin{figure}[h!]
\centering
\begin{tikzpicture}
    \draw[->] (-2.5, 0) -- (2.5,0);
    \draw[->] (0, -2) -- (0, 2);
    \draw[very thick] (0, 0) ellipse (1 and 0.5);
    \draw (1,0) node[anchor=north]{\,\,1};
    \draw (0, 0.4) node[anchor=south]{\footnotesize{\,\,\,\,\,\,\,\,\,\,1/2}};
\end{tikzpicture}
\caption{$\V(I)$}
\end{figure}
\end{column}
\begin{column}{0.48\textwidth}
Let $I:= \gen{(x^2+4y^2-4)^2}$.
$$\I(\V(I)) = \gen{x^2+4y^2-4}.$$ \pause

But how are $\I(\V(I))$ and $I$ related? \pause
$$\boxed{\I(\V(I)) = \sqrt I}$$
\end{column}
\end{columns}
\end{frame}
\begin{frame}{The Nullstellensatz}
\begin{theorem}[Hilbert's Nullstellensatz]
Let $k$ be an algebraically closed field, and let $I\furina k[x_1,\ldots, x_n]$. Then $\I(\V(I)) = \sqrt I$. In particular, if $I$ is radical, then $\I(\V(I)) = I$.
\end{theorem}\pause
Restricting to the set of radical ideals:
$$\Big\{\textsf{algebraic sets in } k^n\Big\} \longleftrightarrow \Big\{\textsf{radical ideals of } k[x_1,\ldots, x_n]\Big\}$$\pause

The field $k$ \textbf{must} be algebraically closed!
\end{frame}
\begin{frame}{The Nullstellensatz, Applications}
\begin{itemize}
    \item ``Fundamental theorem of algebraic geometry" \pause
    \item Nullstellensatz: Shapes $\implies$ abstraction
    \item Abstraction good: equations $\iff$ shapes \pause
    \item Curves in weird fields: $\overline{\F_2}$, etc.
    \item Elliptic curves, number theory, FLT, etc.
\end{itemize}
\end{frame}
\end{document}
